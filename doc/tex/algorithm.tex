\section{Zastosowany algorytm}
Zagadkę rozwiązujemy w dwóch etapach:
\begin{enumerate}
    \item W pierszym etapie eliminujemy podzbiór niedopuszczalnych
          w komórkach piramid iterując po wszystkich ograniczeniach
          na krawędzi zagadki.
    \item W drugim etapie używamy backtrackingu generując częściowe
          rozwiązania przez wstawianie do komórek przefiltrowane
          w pierwszym etapie piramidy.
\end{enumerate}

O ile rozwiązanie zagadki jest możliwe jedynie za pomocą backtrackingu
(lub nawet naiwnego pełnego przeszukiwania), zastosowanie dodatkowego
etapu eliminacji pozwala na obcięcie znacznej części przestrzeni przeszukiwania
a zatem znaczne szybsze znajdowanie rozwiązań dla zagadek nietrywialnych
rozmiarów.

\subsection{Eliminacja}
Eliminacja rozwiązań niedopuszczalnych odbywa się wg następujących reguł:

\begin{itemize}  
  \setlength\itemsep{0.1em}

  \item{W przypadku, gdy w danym wierszu widoczna jest tylko jedna piramida to najwyższa piramida musi stać na pierwszej pozycji. Jeśli mają być widoczne dwie, to najwyższa może stać najwcześniej na drugiej pozycji. W sytuacji gdy ograniczenie widoczności wynosi trzy, to najwyższa może stać najwcześniej na trzeciej, a kolejna od niej niższa na drugiej pozycji licząc od krawędzi na której występuje ograniczenie. Dla kolejnych wartości ograniczeń sposób postępowania jest analogiczny.}

  \item{W sytuacji, gdy w jakiejś komórce jest jednoznaczne rozwiązanie X, to należy wyeliminować wartość X ze wszystkich komórek w wierszu oraz kolumnie do których należy komórka z rozwiązaniem, powtarzając proces rekurencyjnie dla zmienianych komórek.}

\end{itemize}

W kodzie implelmentację eliminacji można znaleźć w module {\texttt Elimination}.

\subsection{Backtracking}
W stworzonej implementacji algorytm znajdowania rozwiązania opiera swoje
działanie na algorytmie z nawrotami (\textit{ang. backtracking}).  W kolejnych
iteracjach rozbudowywane jest częsciowe rozwiązanie do momentu osiągnięcia
pełnej planszy.  Przy tworzeniu kolejnych rozwiązań częściowych sprawdzana
jest ich poprawność pod względem reguł gry.  Jeśli rozwiązanie częściowe
jest niepoprawne, jest ono odrzucane i testujemy kolejną możliwość.  Jeśli
wszystkie możliwe piramidy dla danej komórki zostały odrzucone, wracamy
do poprzedniej komórki i kontynuujemy proces.

W naszym rozwiązaniu zaimplementowaliśmy backtracking w monadzie listy
w sposób inspirowany podejściem opisanym w \cite{backtracking}.

W kodzie implelmentację backtrackingu można znaleźć w module {\texttt Solver}.
